\documentclass[unicode,11pt,a4paper,oneside,numbers=endperiod,openany]{scrartcl}

\input{assignment.sty}

% --- University specific settings ---------------------------------------- %

\usepackage{ifthen}

\newboolean{usi}
\newboolean{eth}

\setboolean{usi}{true}
\setboolean{eth}{false}

%\usieth{usi text}{eth text}
\newcommand{\usieth}[2]
{
	\ifusi #1 \fi
	\ifeth #2 \fi
}

\newcommand{\cluster}
{
	\ifusi ICS cluster \fi
	\ifeth Euler cluster \fi
}

\begin{document}


\setassignment
\setduedate{ Dec 20, 2020, 12pm (midnight)}

\serieheader{\usieth{HPC Lab}{High-Performance Computing Lab for CSE}}{2020}{Student: FULL NAME}{Discussed with: FULL NAME}{Solution for Project 8}{}
\newline


\section*{Project 8 -- Numerical Mathematical Software for Extreme-Scale Science and 
             Interactive Supercomputing with JupyterLab}

\assignmentpolicy

% -------------------------------------------------------------------------- %
% -------------------------------------------------------------------------- %
% --- Exercise 1 ----------------------------------------------------------- %
% -------------------------------------------------------------------------- %
% -------------------------------------------------------------------------- %

\section{Scientific Mathematical HPC Software Frameworks - The Poisson Equation [40 points]}


\begin{table}[h]
	\caption{Wall-clock time (in seconds) and speedup (in brackets) using multiple cores for solving the Poisson PDE problem (adjust cores count according to machined you used to solve the problem).}
	\centering
	
	\medskip
	
	%\footnotesize
	\begin{tabular}{l|r||r|r|r|r}\hline\hline
		Problem & \multicolumn{1}{c||}{$N$} &  \multicolumn{4}{c}{Number cores} \\
		&       & \multicolumn{1}{c|}{1} & \multicolumn{1}{c|}{2} & \multicolumn{1}{c|}{4} & \multicolumn{1}{c}{8} \\
		\hline\hline
		{ Poisson} & $100^2$  &    \phantom{222222}        &    \phantom{222222}      & \phantom{222222}         &      \phantom{222222} \\
		{ Poisson} & $500^2$  &    \phantom{222222}        &    \phantom{222222}      & \phantom{222222}         &      \phantom{222222} \\
		{ Poisson} & $1000^2$ &            &          &          &       \\\hline \hline
	\end{tabular}
	
	\label{tab:PDEparallel1}
\end{table}


% -------------------------------------------------------------------------- %
% -------------------------------------------------------------------------- %
% --- Exercise 2 ----------------------------------------------------------- %
% -------------------------------------------------------------------------- %
% -------------------------------------------------------------------------- %

\section{Interactive Supercomputing using Jupyter Notebook  [10 points]}


% -------------------------------------------------------------------------- %
% -------------------------------------------------------------------------- %
% --- Exercise 3 ----------------------------------------------------------- %
% -------------------------------------------------------------------------- %
% -------------------------------------------------------------------------- %


\section{Jupyter Notebook - Parallel PDE-Constrained Optimization [50 points]}


\begin{table}[h]
	\caption{Wall-clock time (in seconds) and speedup (in brackets) using multiple cores for solving the PDE-constrained optimization problem  (adjust cores count according to machined you used to solve the problem).}
	\centering
	\medskip
	%\footnotesize
	\begin{tabular}{l|r||r|r|r|r}\hline\hline
		Problem & \multicolumn{1}{c||}{$N$} &  \multicolumn{4}{c}{Number of cores} \\
		&       & \multicolumn{1}{c|}{1} & \multicolumn{1}{c|}{2} & \multicolumn{1}{c|}{4} & \multicolumn{1}{c}{8} \\
		\hline\hline
		{ Inverse Poisson} & $100^2$  &    \phantom{222222}        &    \phantom{222222}      & \phantom{222222}         &      \phantom{222222} \\
		{ Inverse  Poisson} & $500^2$ &            &          &          &       \\
		{ Inverse Poisson} & $1000^2$ &            &          &          &       \\\hline \hline
	\end{tabular}
	\label{tab:PDEparallel}
\end{table}




\section*{Additional notes and submission details}
Collect all your source code, results and figures in a Jupyter notebook. Be sure it contains also your name and summary of your answers, results and observations for all exercises. When you are satisfied with you notebook, upload it to together with the PDF version (export the notebook as a PDF file) to \href{https://www.icorsi.ch/course/view.php?id=10049}{\usieth{iCorsi}{Moodle}}.
Submit the source code files (together with your used \texttt{Makefile}) in
an archive file (tar, zip, etc.), and summarize your results and
observations for all exercises by writing an extended Latex report.
Use the Latex template provided on the webpage and upload the Latex summary
as a PDF to \href{https://www.icorsi.ch/course/view.php?id=10049}{\usieth{iCorsi}{Moodle}}.

\begin{itemize}
	\item Your submission should be a gzipped tar archive, formatted like project\_number\_lastname\_firstname.zip or project\_number\_lastname\_firstname.tgz. 
	It should contain
	\begin{itemize}
		\item all the source codes of your solutions;
		\item your write-up with your name  project\_number\_lastname\_firstname.pdf.
	\end{itemize}
	\item Submit your .zip/.tgz through \href{https://www.icorsi.ch/course/view.php?id=10049}{\usieth{iCorsi}{Moodle}}..
\end{itemize}





\end{document}
